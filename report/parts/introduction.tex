\chapter*{\begin{center}
	\vspace{-2.5em}	ВВЕДЕНИЕ \vspace{-1.5em}
	\end{center}
}

\addcontentsline{toc}{chapter}{ВВЕДЕНИЕ}

Многопользовательские игры представляют собой сложные распределенные системы, в которых согласованность данных между участниками играет ключевую роль. Успех таких игр во многом зависит от обеспечения честности игрового процесса, предотвращения читерства и согласования действий между игроками в реальном времени. Однако проблема достижения консенсуса в подобных системах остается актуальной, поскольку читеры и злоумышленники активно ищут способы нарушить целостность данных. Разработка методов достижения консенсуса в таких условиях требует учета специфики игровых процессов, архитектуры систем и существующих угроз.

Целью данной научно-исследовательской работы является сравнение существующих методов достижения консенсуса в многопользовательских играх.

Для достижения цели необходимо решить следующие задачи:
\begin{enumerate}
	\item исследовать основные принципы консенсуса в распределенных системах и их применение в многопользовательских играх;
	\item рассмотреть ключевые методы защиты игрового процесса от читерства и их роль в обеспечении консенсуса;
	\item формализовать процесс достижения консенсуса в контексте многопользовательских игр;
	\item классифицировать существующие методы консенсуса, применимые к игровым системам;
	\item определить основные критерии оценки этих методов в условиях игр;
	\item провести сравнительный анализ методов консенсуса с учетом специфики игровых систем.
\end{enumerate}

