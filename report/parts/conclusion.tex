\chapter*{\begin{center}
		\vspace{-2.5em}	ЗАКЛЮЧЕНИЕ \vspace{-1.5em}
	\end{center}
}
\addcontentsline{toc}{chapter}{ЗАКЛЮЧЕНИЕ}

Цель данной научно-исследователькой работы была достигнута -- было проведено сравнение существующих методов достижения консенсуса в многопользовательских играх.

Для достижения поставленной цели были выполнены следующие задачи:
\begin{enumerate}
	\item исследованы основные принципы консенсуса в распределенных системах и их применение в многопользовательских играх;
	\item рассмотрены ключевые методы защиты игрового процесса от читерства и их роль в обеспечении консенсуса;
	\item формализован процесс достижения консенсуса в контексте многопользовательских игр;
	\item определены основные критерии оценки этих методов в условиях игр;
	\item проведен сравнительный анализ методов консенсуса с учетом специфики игровых систем.
\end{enumerate}

В результате исследования было выявлено, что выбор метода консенсуса для многопользовательских игр зависит от требований конкретной игровой системы:

\begin{itemize}
	\item PBFT показывает высокую устойчивость к атакам, однако его масштабируемость ограничена, что делает его подходящим для игр с высокой степенью доверия и малым количеством участников;
	\item Raft отличается высокой продуктивностью и является оптимальным выбором для систем с умеренными требованиями к масштабируемости;
	\item Paxos демонстрирует сбалансированные характеристики, что позволяет использовать его в сценариях, где требуется высокая масштабируемость и устойчивость к сбоям.
\end{itemize}
